
\chapter{Étude Conceptuelle}
\label{sec:unchapitre}
\fancyhead{} % clear all header fields

\fancyhead[RO,LE]{\textbf{Étude Conceptuelle}}

\fancyfoot{} % clear all footer fields
\clearpage
\fancyfoot[LE,RO]{\thepage}


\section*{Introduction}
\addcontentsline{toc}{section}{Introduction}

 Dans ce chapitre, nous analysons en profondeur les concepts essentiels qui sous-tendent le développement de la solution. L'analyse conceptuelle fera l'objet de ce chapitre en utilisant les diagrammes des cas d'utilisations raffinées , diagramme de classe et quelques diagrammes de séquence.


\section{Diagrammes des cas d'utilisations raffinés}
Pour chaque acteur identifié dans le deuxième chapitre, il est impératif d'explorer en détail les différentes intentions professionnelles qui orientent leur utilisation des systèmes. Commençons par présenter de manière approfondie les cas d'utilisation raffinés.
\begin{figure}[ht!]
  \centering
  \includegraphics[width=12cm,height=8cm]{images/Capture d’écran 2023-05-29 à 18.18.08.png}
  \caption{
   Processus de conception}
\end{figure}

\clearpage

\subsection{xxxxxxxxxxx}
La figure ci dessous décrit le processus de xxxxxx.
\begin{figure}[hbt!]
  \centering
  \includegraphics[width=17cm,height=13cm]{images/ fichier.png}
  \caption{
   Diagramme de cas d'utilisation "xxxxx"}
\end{figure}
\\
Description Description Description Description Description Description Description Description Description Description Description Description Description Description Description Description .

\clearpage
 \subsection{yyyyyyyyyyy}
 La figure ci dessous décrit le yyyyyyyyy.
\begin{figure}[hbt!]
  \centering
  \includegraphics[width=13cm,height=7cm]{images/yyyyyy.png}
  \caption{
   Diagramme de cas d'utilisation "yyyyyyyyy"}
\end{figure}
\\
Description Description Description Description Description Description Description Description Description Description Description Description Description Description Description.

\subsection{zzzzzzzzzzzzzzz}
La figure ci dessous décrit le zzzzzzzzzzz.
\\
\begin{figure}[hbt!]
  \centering
  \includegraphics[width=13cm,height=5cm]{images/Echafichier.png}
  \caption{
   Diagramme de cas d'utilisation "yyyyyyyyy"}
\end{figure}
\\
Description Description Description Description Description Description Description Description Description Description Description Description Description Description Description.



\section{Diagramme de séquences}
Les diagrammes de séquence jouent un rôle crucial dans la modélisation et la conception des systèmes logiciels orientés objet. Ils permettent de représenter visuellement l'ordre des interactions entre les objets et les composants d'un système au fil du temps.
\newline
Le passage de diagrammes de cas d'utilisation vers des diagrammes de séquence permet de détailler comment les interactions entre les acteurs et les cas d'utilisation se déroulent au niveau de la séquence d'actions entre les objets du système.

\begin{figure}[hbt!]
  \centering
\includegraphics[width=14cm,height=14cm]{images/diagrammesequence.png}
  \caption{
   Diagramme de séquence "nnnnnnnnn"}
\end{figure}
Description Description Description Description Description Description Description Description Description Description Description Description Description Description Description.


\clearpage

\section{Diagramme des classes de conception}
Le diagramme de classes de conception se concentre sur la structure interne des classes. Il vise à montrer comment les classes collaborent pour implémenter les fonctionnalités spécifiques du système. Ce type de diagramme est particulièrement utile lors de la phase de conception détaillée.
\begin{figure}[ht!]
  \centering
  \includegraphics[width=12cm,height=8cm]{images/classe.png}
  \caption{
  Diagramme de classes de conception}
\end{figure}
\\

\section*{Conclusion}
\addcontentsline{toc}{section}{Conclusion}

Dans ce chapitre nous avons présenté une conception détaillé a savoir les diagrammes des cas d’utilisation raffinés de xxxxxx, ensuite les diagrammes de séquences et enfin le diagramme de classe. Le prochain chapitre fait l'objet de la partie réalisation du projet.


