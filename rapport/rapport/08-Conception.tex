\chapter{Conception}
\label{chap:conception}

\section{Introduction}

Ce chapitre présente la conception technique de la plateforme Digital Village. Nous détaillerons l'architecture globale, les diagrammes UML (séquence et classes) ainsi que le modèle de données.

\section{Architecture globale}

\subsection{Architecture 3-tiers}

Notre application suit une architecture 3-tiers moderne :

\begin{figure}[H]
\centering
\fbox{\parbox{12cm}{
\textbf{Architecture Digital Village}
\vspace{0.5cm}

\textbf{Couche Présentation (Frontend)}
\begin{itemize}
    \item Next.js 15 (React 19)
    \item Tailwind CSS
    \item TypeScript
\end{itemize}
\vspace{0.3cm}

$\downarrow$ API REST (JSON) $\downarrow$
\vspace{0.3cm}

\textbf{Couche Métier (Backend)}
\begin{itemize}
    \item Node.js + Express
    \item JWT Authentication
    \item Controllers \& Middleware
\end{itemize}
\vspace{0.3cm}

$\downarrow$ Mongoose ODM $\downarrow$
\vspace{0.3cm}

\textbf{Couche Données}
\begin{itemize}
    \item MongoDB (NoSQL)
    \item Collections: Users, Classes, Progress
\end{itemize}
}}
\caption{Architecture 3-tiers}
\end{figure}

\subsection{Pattern MVC}

Le backend suit le pattern MVC (Model-View-Controller) :

\begin{itemize}
    \item \textbf{Models} : Schémas Mongoose (User, Class)
    \item \textbf{Views} : Réponses JSON de l'API
    \item \textbf{Controllers} : Logique métier (authController, progressController, etc.)
\end{itemize}

\section{Diagrammes de séquence}

\subsection{Séquence : Authentification}

\begin{figure}[H]
\centering
\fbox{\parbox{13cm}{
\textbf{Diagramme de séquence - Connexion utilisateur}
\vspace{0.5cm}

\texttt{Utilisateur -> Frontend : Saisie email/password}

\texttt{Frontend -> Backend : POST /api/auth/login}

\texttt{Backend -> MongoDB : findOne(\{email\})}

\texttt{MongoDB -> Backend : User document}

\texttt{Backend -> Backend : bcrypt.compare(password)}

\texttt{Backend -> Backend : jwt.sign(payload)}

\texttt{Backend -> Frontend : \{success, token, user\}}

\texttt{Frontend -> LocalStorage : Stocker token}

\texttt{Frontend -> Utilisateur : Redirection dashboard}
}}
\caption{Séquence d'authentification}
\end{figure}

\subsection{Séquence : Génération de quiz}

\begin{figure}[H]
\centering
\fbox{\parbox{13cm}{
\textbf{Diagramme de séquence - Quiz IA}
\vspace{0.5cm}

\texttt{Utilisateur -> Frontend : Choix sujet + difficulté}

\texttt{Frontend -> Backend : POST /api/quiz/generate}

\texttt{Backend -> Backend : Vérifier OPENAI\_API\_KEY}

\texttt{alt [API Key présente]}

\texttt{~~~~Backend -> OpenAI : Générer questions}

\texttt{~~~~OpenAI -> Backend : Questions JSON}

\texttt{else [Pas d'API Key]}

\texttt{~~~~Backend -> Backend : Utiliser fallback (80+ questions)}

\texttt{end}

\texttt{Backend -> Frontend : \{questions\}}

\texttt{Frontend -> Utilisateur : Afficher quiz}
}}
\caption{Séquence de génération de quiz}
\end{figure}

\section{Diagramme de classes}

\subsection{Modèle User}

\begin{figure}[H]
\centering
\fbox{\parbox{10cm}{
\textbf{User}
\hrule
\vspace{0.2cm}
- \_id: ObjectId \\
- name: String \\
- email: String (unique) \\
- password: String (hashed) \\
- role: Enum [student, teacher, admin, public, super\_admin] \\
- xp: Number (default: 0) \\
- level: Number (default: 1) \\
- badges: Array<Badge> \\
- completedSteps: Array<String> \\
- quizScores: Array<QuizScore> \\
- loginStreak: Number \\
- lastLogin: Date \\
- classId: ObjectId (ref: Class) \\
- createdAt: Date \\
\hrule
\vspace{0.2cm}
+ matchPassword(password): Boolean \\
+ calculateLevel(): Number \\
+ addXP(amount): void \\
+ unlockBadge(badge): void
}}
\caption{Classe User}
\end{figure}

\subsection{Modèle Class}

\begin{figure}[H]
\centering
\fbox{\parbox{10cm}{
\textbf{Class}
\hrule
\vspace{0.2cm}
- \_id: ObjectId \\
- name: String \\
- code: String (unique, 6 chars) \\
- teacher: ObjectId (ref: User) \\
- students: Array<ObjectId> (ref: User) \\
- createdAt: Date \\
\hrule
\vspace{0.2cm}
+ generateCode(): String \\
+ addStudent(userId): void \\
+ removeStudent(userId): void \\
+ getProgress(): Array<Progress>
}}
\caption{Classe Class}
\end{figure}

\section{Modèle de données MongoDB}

\subsection{Collection Users}

\begin{verbatim}
{
  "_id": ObjectId("..."),
  "name": "Mohammed Aziz",
  "email": "aziz@example.com",
  "password": "$2b$10$...", // bcrypt hash
  "role": "student",
  "xp": 450,
  "level": 3,
  "badges": [
    {
      "badgeId": "newcomer",
      "name": "Nouveau Membre",
      "unlockedAt": ISODate("2025-12-05")
    }
  ],
  "completedSteps": ["step1", "step2"],
  "loginStreak": 5,
  "createdAt": ISODate("2025-12-05")
}
\end{verbatim}

\subsection{Collection Classes}

\begin{verbatim}
{
  "_id": ObjectId("..."),
  "name": "Terminale S1",
  "code": "ABC123",
  "teacher": ObjectId("..."), // ref User
  "students": [
    ObjectId("..."),
    ObjectId("...")
  ],
  "createdAt": ISODate("2025-12-05")
}
\end{verbatim}

\section{API REST}

\subsection{Endpoints principaux}

\begin{table}[H]
\centering
\small
\begin{tabular}{|l|l|p{5cm}|}
\hline
\rowcolor{Gray}
\textbf{Méthode} & \textbf{Endpoint} & \textbf{Description} \\
\hline
POST & /api/auth/register & Inscription \\
\hline
POST & /api/auth/login & Connexion \\
\hline
GET & /api/auth/me & Profil utilisateur \\
\hline
POST & /api/quiz/generate & Générer un quiz \\
\hline
GET & /api/progress & Progression utilisateur \\
\hline
POST & /api/progress/step/:id & Compléter une étape \\
\hline
GET & /api/teacher/classes & Classes de l'enseignant \\
\hline
GET & /api/admin/users & Liste des utilisateurs \\
\hline
\end{tabular}
\caption{Endpoints API REST}
\end{table}

\section{Conclusion}

Ce chapitre a présenté l'architecture technique, les diagrammes UML et le modèle de données. Le chapitre suivant détaillera la réalisation concrète et les technologies utilisées.

\clearpage
