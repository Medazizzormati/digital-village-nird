\markboth{Introduction générale}{Introduction générale}
\addcontentsline{toc}{chapter}{Introduction générale}
\label{chap:Introduction générale}
\chapter*{Introduction générale}

\section*{Contexte général}

À l'ère du numérique, les établissements scolaires font face à un défi majeur : la dépendance croissante aux solutions propriétaires des géants du numérique (GAFAM). Cette situation soulève des questions cruciales concernant la souveraineté numérique, la protection des données des élèves et l'impact environnemental des technologies utilisées.

Le collectif \textbf{NIRD} (Numérique Inclusif, Responsable et Durable) propose une démarche innovante pour accompagner les établissements scolaires vers un numérique plus souverain. Cette approche repose sur trois piliers fondamentaux :
\begin{itemize}
    \item \textbf{Inclusif} : Garantir l'accès aux technologies pour tous
    \item \textbf{Responsable} : Protéger les données et assurer une gouvernance transparente
    \item \textbf{Durable} : Promouvoir la sobriété numérique et le réemploi du matériel
\end{itemize}

\section*{Problématique}

La fin du support de Windows 10 prévue pour octobre 2025 va rendre obsolètes des millions d'ordinateurs encore parfaitement fonctionnels. Les établissements scolaires se trouvent confrontés à un choix : investir massivement dans du nouveau matériel ou trouver des alternatives durables.

C'est dans ce contexte que s'inscrit notre projet, réalisé dans le cadre de la \textbf{Nuit de l'Info 2025}, répondant au défi proposé par le collectif NIRD :

\begin{quote}
\textit{"Le Village Numérique Résistant : Comment les établissements scolaires peuvent tenir tête aux Big Tech ? David contre Goliath, Astérix contre l'Empire numérique."}
\end{quote}

\section*{Objectifs du projet}

Notre projet \textbf{Digital Village} vise à créer une plateforme web éducative et interactive permettant de :
\begin{enumerate}
    \item Présenter la démarche NIRD de manière accessible et engageante
    \item Proposer des ressources pédagogiques sur les alternatives libres
    \item Offrir une expérience gamifiée pour motiver l'apprentissage
    \item Accompagner les différents acteurs (élèves, enseignants, administrateurs)
\end{enumerate}

\section*{Organisation du rapport}

Ce rapport est structuré en quatre chapitres :

\begin{itemize}
    \item \textbf{Chapitre 1 - Étude préalable} : Présentation du contexte, analyse de l'existant et méthodologie adoptée
    \item \textbf{Chapitre 2 - Analyse} : Identification des besoins, spécification des exigences et cas d'utilisation
    \item \textbf{Chapitre 3 - Conception} : Architecture technique, diagrammes UML et modélisation de la base de données
    \item \textbf{Chapitre 4 - Réalisation} : Technologies utilisées, interfaces développées et tests effectués
\end{itemize}

\clearpage
