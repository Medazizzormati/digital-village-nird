\chapter{Réalisation}
\label{chap:realisation}

\section{Introduction}

Ce chapitre présente la réalisation concrète du projet Digital Village. Nous détaillerons les technologies utilisées, l'environnement de développement, puis nous présenterons les principales interfaces de l'application.

\section{Technologies utilisées}

\subsection{Frontend}

\begin{table}[H]
\centering
\begin{tabular}{|l|l|p{6cm}|}
\hline
\rowcolor{Gray}
\textbf{Technologie} & \textbf{Version} & \textbf{Utilisation} \\
\hline
Next.js & 15.0 & Framework React avec SSR \\
\hline
React & 19.2 & Bibliothèque UI \\
\hline
TypeScript & 5.x & Typage statique \\
\hline
Tailwind CSS & 4.1 & Framework CSS utility-first \\
\hline
Radix UI & Latest & Composants accessibles \\
\hline
Lucide React & 0.454 & Icônes SVG \\
\hline
\end{tabular}
\caption{Technologies Frontend}
\end{table}

\subsection{Backend}

\begin{table}[H]
\centering
\begin{tabular}{|l|l|p{6cm}|}
\hline
\rowcolor{Gray}
\textbf{Technologie} & \textbf{Version} & \textbf{Utilisation} \\
\hline
Node.js & 20.x & Runtime JavaScript \\
\hline
Express & 4.19 & Framework web \\
\hline
MongoDB & 8.x & Base de données NoSQL \\
\hline
Mongoose & 8.5 & ODM pour MongoDB \\
\hline
JWT & 9.0 & Authentification \\
\hline
Bcrypt & 2.4 & Hachage des mots de passe \\
\hline
\end{tabular}
\caption{Technologies Backend}
\end{table}

\subsection{Outils de développement}

\begin{table}[H]
\centering
\begin{tabular}{|l|p{8cm}|}
\hline
\rowcolor{Gray}
\textbf{Outil} & \textbf{Utilisation} \\
\hline
Visual Studio Code & Éditeur de code principal \\
\hline
Git / GitHub & Gestion de versions \\
\hline
Vercel & Déploiement frontend \\
\hline
Postman & Tests API \\
\hline
npm & Gestionnaire de paquets \\
\hline
\end{tabular}
\caption{Outils de développement}
\end{table}

\section{Structure du projet}

\subsection{Arborescence Frontend}

\begin{verbatim}
digital-village-nird/
├── app/                    # Pages Next.js (App Router)
│   ├── page.tsx           # Page d'accueil
│   ├── login/             # Page de connexion
│   ├── signup/            # Page d'inscription
│   ├── dashboard/         # Tableau de bord
│   ├── quiz/              # Quiz classique
│   ├── ai-quiz/           # Quiz IA
│   ├── resources/         # Ressources
│   └── api/               # Routes API Next.js
├── components/            # Composants réutilisables
│   ├── navigation.tsx     # Barre de navigation
│   ├── footer.tsx         # Pied de page
│   ├── media-card.tsx     # Carte média
│   └── video-modal.tsx    # Modal vidéo
├── lib/                   # Utilitaires
│   └── api.ts            # Client API
└── public/               # Assets statiques
\end{verbatim}

\subsection{Arborescence Backend}

\begin{verbatim}
backend/
├── src/
│   ├── server.js          # Point d'entrée
│   ├── config/
│   │   └── db.js         # Connexion MongoDB
│   ├── models/
│   │   ├── User.js       # Modèle utilisateur
│   │   └── Class.js      # Modèle classe
│   ├── controllers/
│   │   ├── authController.js
│   │   ├── progressController.js
│   │   ├── teacherController.js
│   │   └── adminController.js
│   ├── middleware/
│   │   └── auth.js       # Middleware JWT
│   └── routes/
│       ├── auth.js
│       ├── progress.js
│       ├── teacher.js
│       └── admin.js
└── package.json
\end{verbatim}

\section{Interfaces de l'application}

\subsection{Page d'accueil}

La page d'accueil présente Digital Village avec un design gaming professionnel :
\begin{itemize}
    \item Hero section avec animation de grille
    \item Badge "Nuit de l'Info 2025"
    \item Titre bilingue (Arabe/Français)
    \item Boutons CTA (Commencer, Quiz)
    \item Statistiques animées (5 étapes, 50+ questions)
    \item Présentation des 3 piliers NIRD
\end{itemize}

\subsection{Pages d'authentification}

Les pages de connexion et d'inscription présentent un design split-screen moderne :
\begin{itemize}
    \item Panneau gauche : Branding et informations
    \item Panneau droit : Formulaire
    \item Toggle Mode API / Mode Démo
    \item Sélection de rôle avec icônes (inscription)
    \item Validation en temps réel
    \item Effets visuels (glow, gradients)
\end{itemize}

\subsection{Dashboard utilisateur}

Le tableau de bord affiche :
\begin{itemize}
    \item Avatar dynamique avec couleur selon le rôle
    \item Niveau et barre de progression XP
    \item Compteur de streak
    \item 4 cartes de statistiques animées
    \item Progression en 5 étapes
    \item 6 badges avec niveaux de rareté
    \item Actions rapides
    \item Activité récente
\end{itemize}

\subsection{Quiz IA}

La page de quiz interactif propose :
\begin{itemize}
    \item 8 sujets suggérés avec icônes
    \item Sélection de difficulté (Facile, Moyen, Difficile)
    \item Choix du nombre de questions (1-10)
    \item Génération par OpenAI ou fallback local
    \item Affichage progressif des questions
    \item Feedback immédiat avec explications
    \item Score final avec encouragements
\end{itemize}

\subsection{Page Ressources}

La bibliothèque de ressources contient :
\begin{itemize}
    \item 3 catégories (Documentation, Vidéos, Outils)
    \item Cartes de ressources avec tags
    \item Lecteur vidéo intégré (modal)
    \item Support YouTube, Vimeo, PeerTube
    \item Section Médias avec couverture presse
    \item Liens rapides thématiques
\end{itemize}

\section{Fonctionnalités clés implémentées}

\subsection{Système de gamification}

\begin{itemize}
    \item \textbf{XP (Experience Points)} : Gagnés à chaque action (quiz, étapes)
    \item \textbf{Niveaux} : Calculés selon la formule $niveau = \lfloor XP / 200 \rfloor + 1$
    \item \textbf{Badges} : 6 badges avec raretés (Common, Rare, Epic, Legendary)
    \item \textbf{Streaks} : Jours consécutifs de connexion
\end{itemize}

\subsection{Quiz intelligent}

Le système de quiz utilise une approche hybride :
\begin{enumerate}
    \item Si OpenAI API configurée : génération dynamique
    \item Sinon : utilisation de 80+ questions prédéfinies
    \item 8 thèmes : Linux, Sécurité, Environnement, Logiciels Libres, Inclusion, Éducation, Vie Privée, Matériel
\end{enumerate}

\subsection{Lecteur vidéo intégré}

Le composant VideoModal permet de lire des vidéos sans quitter le site :
\begin{itemize}
    \item Extraction automatique des IDs YouTube/Vimeo
    \item Support PeerTube (Tube éducatif)
    \item Fallback vers lien externe si non supporté
    \item Fermeture par clic extérieur ou touche Échap
\end{itemize}

\section{Déploiement}

\subsection{Frontend sur Vercel}

Le déploiement sur Vercel est automatisé :
\begin{enumerate}
    \item Push sur la branche main de GitHub
    \item Détection automatique par Vercel
    \item Build Next.js
    \item Déploiement sur CDN global
\end{enumerate}

\subsection{Variables d'environnement}

\begin{verbatim}
# Frontend (optionnel)
OPENAI_API_KEY=sk-...

# Backend
PORT=5000
JWT_SECRET=secret_key
MONGO_URI=mongodb://localhost:27017/nird_db
\end{verbatim}

\section{Tests effectués}

\begin{table}[H]
\centering
\begin{tabular}{|l|l|l|}
\hline
\rowcolor{Gray}
\textbf{Type} & \textbf{Élément testé} & \textbf{Résultat} \\
\hline
Fonctionnel & Inscription/Connexion & OK \\
\hline
Fonctionnel & Quiz IA (avec/sans API) & OK \\
\hline
Fonctionnel & Lecteur vidéo & OK \\
\hline
UI/UX & Responsive design & OK \\
\hline
Performance & Temps de chargement & < 2s \\
\hline
Sécurité & JWT Authentication & OK \\
\hline
\end{tabular}
\caption{Résultats des tests}
\end{table}

\section{Conclusion}

Ce chapitre a présenté la réalisation technique du projet Digital Village, les technologies utilisées et les principales interfaces. Le projet est fonctionnel et déployable sur Vercel.

\clearpage
