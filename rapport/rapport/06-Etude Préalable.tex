\chapter{Étude préalable}
\label{chap:etude}

\section{Introduction}

Ce premier chapitre présente le cadre général du projet Digital Village. Nous commencerons par présenter le contexte de la Nuit de l'Info 2025 et le défi NIRD, puis nous analyserons les solutions existantes avant de présenter notre approche et la méthodologie adoptée.

\section{Présentation du contexte}

\subsection{La Nuit de l'Info 2025}

La Nuit de l'Info est un défi national qui rassemble chaque année des milliers d'étudiants, enseignants et développeurs autour de défis informatiques. L'édition 2025, qui s'est déroulée du jeudi 4 décembre à 16h34 au vendredi 5 décembre à 8h04, a proposé plusieurs sujets dont celui du collectif NIRD.

\subsection{Le collectif NIRD}

Le collectif NIRD (Numérique Inclusif, Responsable et Durable) est une initiative portée par des enseignants et professionnels du numérique. Leur objectif est de promouvoir des pratiques numériques respectueuses de l'environnement, inclusives et souveraines dans le milieu éducatif.

\subsection{Le défi proposé}

Le sujet intitulé \textit{"Le Village Numérique Résistant"} nous demandait de créer une plateforme pour :
\begin{itemize}
    \item Aider les établissements scolaires à réduire leurs dépendances numériques
    \item Promouvoir les alternatives libres (Linux, LibreOffice, etc.)
    \item Sensibiliser à la sobriété numérique
    \item Accompagner la transition vers un numérique souverain
\end{itemize}

\section{Analyse de l'existant}

\subsection{Solutions actuelles}

Actuellement, plusieurs ressources existent pour promouvoir le logiciel libre dans l'éducation :

\begin{table}[H]
\centering
\begin{tabular}{|l|p{8cm}|}
\hline
\rowcolor{Gray}
\textbf{Solution} & \textbf{Description} \\
\hline
Site NIRD & Site officiel présentant la démarche (statique) \\
\hline
Primtux & Distribution Linux pour les écoles primaires \\
\hline
Framasoft & Association promouvant les logiciels libres \\
\hline
APRIL & Association pour la promotion du logiciel libre \\
\hline
\end{tabular}
\caption{Solutions existantes}
\end{table}

\subsection{Limites identifiées}

Les solutions actuelles présentent plusieurs limites :
\begin{itemize}
    \item Manque d'interactivité et d'engagement
    \item Absence de parcours pédagogique structuré
    \item Pas de système de progression ou de gamification
    \item Contenu parfois trop technique pour les débutants
    \item Pas de gestion des différents profils utilisateurs
\end{itemize}

\section{Solution proposée}

\subsection{Présentation de Digital Village}

Notre solution, \textbf{Digital Village}, est une plateforme web moderne et interactive qui répond aux limites identifiées :

\begin{itemize}
    \item \textbf{Expérience gamifiée} : Système de points (XP), niveaux, badges et streaks
    \item \textbf{Parcours structuré} : 5 étapes progressives de la démarche NIRD
    \item \textbf{Quiz interactif} : Génération de questions par IA sur 8 thèmes
    \item \textbf{Ressources multimédias} : Vidéos, PDF, tutoriels intégrés
    \item \textbf{Multi-profils} : Élève, Enseignant, Administrateur, Public
\end{itemize}

\subsection{Fonctionnalités principales}

\begin{enumerate}
    \item \textbf{Page d'accueil} : Présentation attractive avec animations
    \item \textbf{Parcours NIRD} : 5 étapes interactives avec progression
    \item \textbf{Quiz IA} : 80+ questions générées dynamiquement
    \item \textbf{Bibliothèque} : Ressources documentaires et vidéos
    \item \textbf{Dashboard} : Tableau de bord personnalisé
    \item \textbf{Authentification} : Inscription/connexion sécurisée
\end{enumerate}

\section{Méthodologie adoptée}

\subsection{Approche Agile}

Compte tenu du temps limité (une nuit), nous avons adopté une approche Agile simplifiée avec des sprints courts :

\begin{itemize}
    \item \textbf{Sprint 1 (2h)} : Setup du projet et page d'accueil
    \item \textbf{Sprint 2 (3h)} : Authentification et dashboard
    \item \textbf{Sprint 3 (3h)} : Quiz et ressources
    \item \textbf{Sprint 4 (2h)} : Backend et API
    \item \textbf{Sprint 5 (2h)} : Tests et déploiement
\end{itemize}

\subsection{Outils de collaboration}

\begin{itemize}
    \item \textbf{GitHub} : Gestion du code source
    \item \textbf{Discord} : Communication en temps réel
    \item \textbf{Figma} : Maquettes UI/UX
    \item \textbf{VS Code} : Environnement de développement
\end{itemize}

\section{Conclusion}

Ce chapitre a permis de présenter le contexte du projet, d'analyser les solutions existantes et de définir notre approche. Le chapitre suivant détaillera l'analyse des besoins et les spécifications fonctionnelles de notre solution.

\clearpage
