\chapter{Analyse des besoins}
\label{chap:analyse}

\section{Introduction}

Ce chapitre est consacré à l'analyse fonctionnelle et technique du projet Digital Village. Nous identifierons les différents acteurs du système, leurs besoins, puis nous présenterons les cas d'utilisation et les exigences fonctionnelles.

\section{Identification des acteurs}

Notre plateforme s'adresse à plusieurs types d'utilisateurs :

\begin{table}[H]
\centering
\begin{tabular}{|l|p{9cm}|}
\hline
\rowcolor{Gray}
\textbf{Acteur} & \textbf{Description} \\
\hline
Visiteur & Utilisateur non authentifié qui découvre la plateforme \\
\hline
Élève (Student) & Apprenant qui suit le parcours NIRD et passe les quiz \\
\hline
Enseignant (Teacher) & Formateur qui gère ses classes et suit la progression \\
\hline
Administrateur & Gestionnaire de la plateforme et des utilisateurs \\
\hline
Super Admin & Accès complet à toutes les fonctionnalités \\
\hline
\end{tabular}
\caption{Acteurs du système}
\end{table}

\section{Besoins fonctionnels}

\subsection{Besoins du visiteur}
\begin{itemize}
    \item Consulter la page d'accueil et découvrir NIRD
    \item Accéder aux ressources publiques
    \item S'inscrire sur la plateforme
    \item Se connecter à son compte
\end{itemize}

\subsection{Besoins de l'élève}
\begin{itemize}
    \item Suivre le parcours NIRD en 5 étapes
    \item Passer des quiz et gagner des points XP
    \item Consulter son tableau de bord (progression, badges, niveau)
    \item Accéder à la bibliothèque de ressources
    \item Rejoindre une classe (code d'invitation)
    \item Modifier son profil
\end{itemize}

\subsection{Besoins de l'enseignant}
\begin{itemize}
    \item Créer et gérer des classes
    \item Inviter des élèves dans ses classes
    \item Suivre la progression de ses élèves
    \item Consulter les statistiques de sa classe
    \item Exporter les données de progression
\end{itemize}

\subsection{Besoins de l'administrateur}
\begin{itemize}
    \item Gérer les utilisateurs (CRUD)
    \item Modifier les rôles des utilisateurs
    \item Consulter les statistiques globales
    \item Superviser l'activité de la plateforme
\end{itemize}

\section{Besoins non fonctionnels}

\begin{table}[H]
\centering
\begin{tabular}{|l|p{9cm}|}
\hline
\rowcolor{Gray}
\textbf{Exigence} & \textbf{Description} \\
\hline
Performance & Temps de réponse < 2 secondes \\
\hline
Sécurité & Authentification JWT, mots de passe hashés (bcrypt) \\
\hline
Accessibilité & Interface responsive, compatible mobile \\
\hline
Disponibilité & Application déployée sur Vercel (99.9\% uptime) \\
\hline
Maintenabilité & Code modulaire, documentation complète \\
\hline
Scalabilité & Architecture permettant la montée en charge \\
\hline
\end{tabular}
\caption{Exigences non fonctionnelles}
\end{table}

\section{Cas d'utilisation}

\subsection{Diagramme de cas d'utilisation global}

Le diagramme suivant présente les principales fonctionnalités accessibles à chaque acteur :

\begin{figure}[H]
\centering
\fbox{\parbox{12cm}{
\textbf{Cas d'utilisation - Digital Village}
\vspace{0.5cm}

\textbf{Visiteur :}
\begin{itemize}
    \item Consulter l'accueil
    \item S'inscrire
    \item Se connecter
\end{itemize}

\textbf{Élève :}
\begin{itemize}
    \item Suivre le parcours NIRD
    \item Passer les quiz
    \item Consulter le dashboard
    \item Gérer son profil
\end{itemize}

\textbf{Enseignant :}
\begin{itemize}
    \item Gérer les classes
    \item Suivre les élèves
    \item Exporter les données
\end{itemize}

\textbf{Administrateur :}
\begin{itemize}
    \item Gérer les utilisateurs
    \item Consulter les statistiques
    \item Superviser la plateforme
\end{itemize}
}}
\caption{Cas d'utilisation globaux}
\end{figure}

\subsection{Description des cas d'utilisation principaux}

\subsubsection{CU01 : S'inscrire}
\begin{table}[H]
\centering
\begin{tabular}{|l|p{9cm}|}
\hline
\textbf{Nom} & S'inscrire \\
\hline
\textbf{Acteur} & Visiteur \\
\hline
\textbf{Précondition} & L'utilisateur n'a pas de compte \\
\hline
\textbf{Scénario principal} & 
1. L'utilisateur accède à la page d'inscription \newline
2. Il remplit le formulaire (nom, email, mot de passe, rôle) \newline
3. Il valide le formulaire \newline
4. Le système crée le compte et redirige vers le dashboard \\
\hline
\textbf{Postcondition} & Le compte est créé, l'utilisateur reçoit 100 XP de bonus \\
\hline
\end{tabular}
\caption{Cas d'utilisation : S'inscrire}
\end{table}

\subsubsection{CU02 : Passer un quiz}
\begin{table}[H]
\centering
\begin{tabular}{|l|p{9cm}|}
\hline
\textbf{Nom} & Passer un quiz \\
\hline
\textbf{Acteur} & Élève (authentifié) \\
\hline
\textbf{Précondition} & L'utilisateur est connecté \\
\hline
\textbf{Scénario principal} & 
1. L'utilisateur choisit un sujet de quiz \newline
2. Il sélectionne la difficulté et le nombre de questions \newline
3. Le système génère les questions (IA ou fallback) \newline
4. L'utilisateur répond aux questions \newline
5. Le système affiche le score et les explications \\
\hline
\textbf{Postcondition} & Le score est enregistré, XP attribués \\
\hline
\end{tabular}
\caption{Cas d'utilisation : Passer un quiz}
\end{table}

\section{Conclusion}

Ce chapitre a permis d'identifier les acteurs, leurs besoins et les principales fonctionnalités de la plateforme. Le chapitre suivant présentera la conception technique et les diagrammes UML détaillés.

\clearpage
