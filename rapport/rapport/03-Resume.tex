
\chapter*{Résumé}
\addcontentsline{toc}{chapter}{Résumé}

\section*{Résumé en français}

Ce projet, réalisé dans le cadre de la \textbf{Nuit de l'Info 2025}, répond au défi proposé par le collectif NIRD : créer une plateforme web éducative pour accompagner les établissements scolaires vers un numérique plus souverain, inclusif et durable.

\textbf{Digital Village} est une application web moderne qui présente la démarche NIRD à travers une expérience interactive et gamifiée. Elle propose :
\begin{itemize}
    \item Une présentation complète des 5 étapes de la démarche NIRD
    \item Un système de quiz interactif avec génération de questions par IA
    \item Une bibliothèque de ressources (documentations, vidéos, tutoriels)
    \item Un tableau de bord utilisateur avec système de progression et badges
    \item Une authentification sécurisée avec gestion des rôles (élève, enseignant, admin)
\end{itemize}

Le projet utilise des technologies modernes : Next.js 15 pour le frontend, Node.js/Express pour le backend, et MongoDB pour la base de données. L'interface adopte un design gaming professionnel avec des effets visuels soignés.

\textbf{Mots-clés :} NIRD, numérique responsable, logiciels libres, Linux, éducation, gamification, Next.js, Node.js

\vspace{1cm}

\section*{Abstract (English)}

This project, developed during the \textbf{Nuit de l'Info 2025}, addresses the challenge proposed by the NIRD collective: creating an educational web platform to guide schools towards more sovereign, inclusive, and sustainable digital practices.

\textbf{Digital Village} is a modern web application that presents the NIRD approach through an interactive and gamified experience. It offers:
\begin{itemize}
    \item A complete presentation of the 5 steps of the NIRD approach
    \item An interactive quiz system with AI-generated questions
    \item A resource library (documentation, videos, tutorials)
    \item A user dashboard with progression system and badges
    \item Secure authentication with role management (student, teacher, admin)
\end{itemize}

The project uses modern technologies: Next.js 15 for the frontend, Node.js/Express for the backend, and MongoDB for the database. The interface features a professional gaming design with polished visual effects.

\textbf{Keywords:} NIRD, responsible digital, free software, Linux, education, gamification, Next.js, Node.js

\clearpage
